\chapter{非线性方程(组)求解}
\begin{introduction}
\item 求解函数fzero、fslove、roots
\item 数值求解方法
\end{introduction}
\section{求解函数}\label{func}
\subsection{fzero函数}
fzero 函数是 MATLAB 里用于求解\textcolor{third}{\textbf{单个}}非线性方程的函数。它的基本使用格式如下:
\begin{lstlisting}[frame=single,numbers=left]
% 完整形式
[x,fval,exitflag,output]=fzero(fun,x0,options, p1, p2, ...)
\end{lstlisting}

\begin{lstlisting}[frame=single,numbers=left]
% 常用形式
x=fzero(`\mlplaceholder{fun}`,`\mlplaceholder{x0}`)
% 其中,`\mlplaceholder{fun}`为匿名函数或函数句柄
% `\mlplaceholder{x0}`为迭代初值,
% 也可为区间[x1,x2],函数值在端点处异号,在端点内求根
% x为x0附近的解,如有多根,则与x0选取有关
\end{lstlisting}

对于`\mlplaceholder{fun}`的使用举例如下:

\begin{lstlisting}
% 匿名函数
fun1=@(x) x-1;
fzero(fun1,0) %结果为1
\end{lstlisting}

\begin{lstlisting}
% 函数句柄
% 编写.m文件(函数文件也可)
fzero(@fun2,0) %结果为1
function y=fun2(x)
y=x-1;
end %脚本中的所有函数都必须以 'end' 结束,函数(文件)中则不需要
\end{lstlisting}

二者实现的结果类似。

\newpage
\subsection{fsolve函数}
fsolve 函数是 MATLAB 里用于求解非线性\textcolor{third}{\textbf{方程组}}的函数。它的基本使用格式如下:
\begin{lstlisting}[frame=single,numbers=left]
% 完整形式
[x,fval,exit]=fsolve(fun,x0,option)
\end{lstlisting}

对于求解方程组\footnote{摘自上机材料3}:

\[ y =
\begin{cases}
x^2+y^2-1=0\\
0.75x^3-y+0.9=0
\end{cases} \]

可新建函数文件求解:

\begin{lstlisting}
% 新建文件ExperFsolve.m
function ExperFsolve
x0=[0 0];
[x,fval,exit]=fsolve(@fun,x0)
function y=fun(x)
y=zeros(2,1);
y(1)=x(1)^2+x(2)^2-1;
y(2)=0.75*x(1)^3-x(2)+0.9;
\end{lstlisting}

也可使用匿名函数求解,二者运行结果相同:
\begin{lstlisting}
x0=[0 0]; 
f=@(x) [x(1)^2+x(2)^2-1;0.75*x(1)^3-x(2)+0.9];
[x,fval,exit]=fsolve(f,x0)
\end{lstlisting}

\begin{note}
fzero函数和fsolve函数的运行结果为初值附近的解,而非全部解。
\end{note}
\subsection{roots函数}\label{ployp}
fsolve 函数是 MATLAB 里用于求解\textcolor{third}{\textbf{多项式}}的函数。它的基本使用格式如下:
\begin{lstlisting}[frame=single,numbers=left]
r=roots(`\mlplaceholder{p}`)
% 其中`\mlplaceholder{p}`为n次多项式行向量,从高(n次)到低(0次)排列
\end{lstlisting}
譬如,求解方程$x^2-1=0$,
\begin{lstlisting}
roots([1 0 -1]) %结果为[-1;1]
\end{lstlisting}
\begin{note}
roots函数可以获得多项式的所有根。
\end{note}

\newpage
\section{常用数值求解方法}
非线性方程求数值解方法有:

\textbf{逐步扫描法,二分法,牛顿法,割线法,逆二次插值...}

\begin{definition}{二分法}{}
\textbf{[基本思想]}\\
区间逐次二等分至精度要求。

\textbf{[求解精度]}\\
可靠,只要次数足够多一定能到指定精度。

\textbf{[效率]}\\
收敛速度较慢。
\end{definition}

\begin{proposition}{收敛性对比}{}
\textbf{[牛顿法]}\\
二次收敛;收敛速度不稳定,某些条件下收敛慢或根本不收敛;导数计算不便。

\textbf{[截弦法]}\\
超线性收敛;类似牛顿法;无需计算导数。

\textbf{[逆二次插值]}\\
三点构造抛物线交坐标轴迭代;整个过程收敛速度不稳定;接近终点时迭代快。
\end{proposition}