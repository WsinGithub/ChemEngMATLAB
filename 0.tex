\chapter*{绪论部分}
\addcontentsline{toc}{chapter}{绪论部分}
\markboth{绪论部分}{}
\begin{introduction}
\item 误差
\item 浮点数及其运算
\item MATLAB 的常用命令
\end{introduction}
\section{误差}
\subsection{误差来源}
\begin{itemize}
  \item \textbf{模型误差},问题简化过程产生
  \item \textbf{截断误差},有限次运算限制产生
  \item \textbf{舍入误差},机器字长限制产生
\end{itemize}
有关误差定义不再赘述,举Work1中一例:
\begin{problem}
已知某化工管道的真实长度为 1000m,某次测量结果为 1001m,其测量的绝
对误差,相对误差为多少?
\end{problem}

\begin{solution}
绝对误差为 1m,相对误差为 0.1\%。
\end{solution}

\section{浮点数与浮点数运算}
\subsection{浮点数}
由于计算机资源的有限,在计算机上只能表示有限的实数,这些数被称为浮点数。

浮点数有如下性质:\textbf{有限个、有界、非连续。}
\subsection{IEEE标准双精度浮点运算体系}

\begin{definition}{特殊浮点数}{}
最大实数(上溢:超过用inf或Inf表示):
\[real_{max}=1.79e+308
\]
最小正实数(下溢:若为小于其的正数,则记为0):
\[real_{min}=2.225e-308
\]
从1到下一个较大浮点数的距离(机器精度):
\[eps=2^{-52}=2.220e-16
\]
\end{definition}
\subsection{浮点数运算}
浮点数运算,加法和乘法运算交换律仍然适用,但是
其结合律和分配律已不再适用。
\begin{definition}{NaN}{}
 Not a Number,非数。
\end{definition}

计算下列情况时会出现,

\begin{itemize}
  \item 0 / 0
  \item $\infty / \infty$
  \item (+Inf) + (-Inf)
  \item 0 * Inf
\end{itemize}
\section{MATLAB的通用命令}
\begin{definition}{常用命令}{}
 clc:清除命令窗口内容
 
 clear:清除内存变量
 
 save:保存内存变量到指定文件
\end{definition}

\begin{lstlisting}[frame=single,numbers=left]
clc,clear %通常用作初始化
\end{lstlisting}