\chapter*{写在前面\footnote{点击目录后的页码可快速跳转到指定内容,也可以在书签页快速浏览。}}
期末临近,本文将作为华理化工学院专业必修课程“计算机化工应用”的复习指南。

本文制作过程中主要参考了隋志军老师的教材《化工数值计算与MATLAB》、作业、课件,并参考了付金硕学长的相关资料。在文档制作过程中还得到了\LaTeX{}技术交流 1 群与\LaTeX{}科技排版工作室中网(da)友(lao)们的帮助,在此感谢他们!
\begin{lstlisting}[frame=single,numbers=left]
% 此环境下主要介绍函数的写法
其中,`\mlplaceholder{空缺}`通常表示占位符。
while `\mlplaceholder{condition}`
    if `\mlplaceholder{something-happens}`
        % do something useful
    end
end
\end{lstlisting}

\begin{lstlisting}
% 此环境下主要给出例题的解答,各位可以直接复制到MATLAB中运行
function exam5_3_1
t=3:3:30;
G1=[10.2 13.9 12.16 13.49 13.74 12.01 11.55 11.02 12 12.12];
G2=[6.91 8.94 8.69 10.09 10.63 9.58 9.53 9.29 10.13 10.24];
A=1.83;
X=(G1-G2)./G2;
tt=linspace(3,30);
% 插值
pp=spline(t,X);
ppv=-fnval(pp,tt)/A;
% 微分
dp=fnder(pp);dpv=fnval(dp,tt);
plot(t,X,'ro',tt,ppv,tt,dpv,'.-')
axisX=refline(0,0)
axisX.Color='c'
legend('原始数据','拟合曲线','导数曲线','导数参考直线')
\end{lstlisting}

时间仓促,错误\footnote{各位如果发现写错的地方,请联系我QQ568365675!}与疏漏\footnote{后续更新于\href{https://github.com/WsinGithub/ChemEngMATLAB}{https://github.com/WsinGithub/ChemEngMATLAB}}在所难免,请各位辩证地使用!

最后,预祝各位期末顺利,门门4.0!
