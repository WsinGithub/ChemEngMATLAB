\chapter{数值微分与积分}
\begin{introduction}
\item 数值微分的三种思路
\item diff与fnder函数
\item 插值型积分公式
\item quad与quadl函数
\end{introduction}
\section{数值微分}
\subsection{建立数值微分公式}
\begin{definition}{三种思路}{}
[差分]

从微分定义出发,通过近似处理,得到数值微分的近似公式。

[插值]

从插值近似公式出发,对插值公式的近似求导可得到数值微分的近似公式。

[拟合]

先用最小二乘拟合方法根据已知数据或得近似函数(如样条函数),再对此近似函数求微分可得到数值微分的近似公式。
\end{definition}
\subsection{向前差分及diff函数}
在MATLAB中,可用diff求向量相邻元素的差值:
\begin{lstlisting}[frame=single,numbers=left]
% 向前差分函数diff
% 调用形式
Y=diff(X,n,dim)
% X:输入矩阵
% n:差分阶数,默认1
% dim:行1列2,默认按行差分
% Y=diff(X,2)等价于Y=diff(diff(X))
\end{lstlisting}

\begin{lstlisting}[frame=single,numbers=left]
% 常用形式
Y=diff(X)
% 若X为m阶向量,则上式等价于
Y=[X(2)-X(1) X(3)-X(2)..X(m)-X(m-1)]
\end{lstlisting}

\newpage

则可用\textcolor{third}{\textbf{一阶向前差分}}近似计算$\frac{\mathrm{d}x}{\mathrm{d}y}$。
\begin{lstlisting}[frame=single,numbers=left]
diff(y)./diff(x)
\end{lstlisting}



\subsection{三次插值与fnder函数}\label{cubic}
\begin{note}
关于spline/phchip的函数介绍见\ref{spline},不再赘述。
\end{note}

解决该类问题需要用到fnder函数,介绍如下:

\begin{lstlisting}[frame=single,numbers=left]
% Differentiate function
% 调用格式
fprime=fnder(f,dorder) %dorder求导阶数,默认为1
\end{lstlisting}
\begin{lstlisting}[frame=single,numbers=left]
% 常用形式
pp=spline(x,y); %pp为三次样条插值函数
fprime=fnder(pp); %fprime为pp的一阶导函数
value=fnval(fprime,x) %计算在x处的导数值
\end{lstlisting}

\subsection{拟合与微分}\label{fitd}
\begin{note}
此节\ref{fitd}隋志军老师于考试要点W12中未作要求,部分内容可参考拟合章节\ref{fit}。
\end{note}

\begin{lstlisting}[frame=single,numbers=left]
% 多项式拟合求微分
p=ployfit(x,y,n); %多项式向量
dp=polyder(p); %多项式导函数
dpv=ployval(dp,x); %计算函数值,也可用fnval(dp,x)
\end{lstlisting}

\begin{lstlisting}[frame=single,numbers=left]
% 样条拟合求微分
% 可用cspas/spaps/spap2
% 举cspas为例:
pp=csaps(x,y); %pp为样条拟合函数
fprime=fnder(pp); %fprime为pp的一阶导函数
value=fnval(fprime,x) %计算在x处的导数值
\end{lstlisting}
\section{数值积分}
\subsection{基本思路与方法}
基本思路来自于\textcolor{third}{\textbf{插值法}}, 通过构造一个插值多项式$Pn(x)$作为$f(x)$的近似表达式,用$Pn(x)$的积分值作为$f(x)$的近似积分值。
\begin{definition}{Newton-Cotes求积公式}{}
a)等距节点,分为n份;

b)使用Lagrange插值多项式近似。
\end{definition}
特别地,改变其中等分数n=1,2,可得:
\begin{enumerate}
  \item \textbf{梯形求积公式};
  \item \textbf{Simpson求积公式}。
\end{enumerate}

\begin{definition}{复化求积}{}
a)将积分区间$[a,b]$分为n个相等子区间;

b)对每个子区间使用梯形求积或Simpson求积公式。
\end{definition}

\begin{definition}{自适应求积}{}
a)考虑区间上及其二等分以后的两个Simpson积分和,分别记为S1、S2;

b)~S1、S2满足精度要求则不再二等分,取该区间积分值为S2。

[注]:可以不等步长。
\end{definition}

\begin{note}
牛顿-科特斯和高斯-勒让德求积公式,从略。
\end{note}

\subsection{quad和quadl函数}
自适应Simpson法数值积分:\textcolor{third}{\textbf{quad}}。

\begin{lstlisting}[frame=single,numbers=left]
% 调用形式
q=quad(fun,a,b,tol,trace,p1,p2,…)
% fun 被积函数
% a,b 积分上下限
% tol 求解精度
\end{lstlisting}

\begin{lstlisting}[frame=single,numbers=left]
% 常用形式
q=quad(fun,a,b)
\end{lstlisting}
自适应Lobatto数值积分:\textcolor{third}{\textbf{quadl}},结果更精确。
\begin{lstlisting}[frame=single,numbers=left]
% quadl函数与quad函数的使用完全一致
q=quadl(fun,a,b)
q=quadl(fun,a,b,tol,trace,p1,p2,…)
\end{lstlisting}

\begin{note}
被积函数支持向量化运算,函数表达式 中的必须使用点运算符号!
\end{note}

\begin{note}
fun用法请参考\ref{func}。
\end{note}
